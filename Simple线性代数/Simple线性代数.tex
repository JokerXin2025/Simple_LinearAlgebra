
% 2025 Simple·System

% < Simple 线性代数 >

% --------------------------------------------------
% | 导言区 | Preamble (XeLaTeX) |
% --------------------------------------------------

% 宏包 | Packages

\documentclass[10pt]{article}
\usepackage{simplesystem,datetime2,geometry,ctex,amssymb,bm,ulem}

% 编译时间 | Compile Time

\DTMsetstyle{iso}
\special{
    pdf: encrypt
    ownerpw (\DTMtoday)
    length (128)
    perm (0)
}

% 页面规格 | Page Layout

\renewcommand \baselinestretch {1.75}
\newcommand \gap {\hspace{0.15em}}
\xeCJKsetup {
    CJKecglue = \gap
}
\geometry {
    b5paper,
    left=18.75mm,
    right=18.75mm,
    top=20mm,
    bottom=25mm
}

% 字体 | Fonts

\newCJKfontfamily\HeavyFontC{方正春晚龙行体 简 Heavy}
\newCJKfontfamily\BoldFontC{方正春晚龙行体 简 Bold}
\newCJKfontfamily\ThinFontC{方正春晚龙行体 简 Thin}
\newfontfamily\HeavyFontE{FZChunWanLXTS Heavy}

% 杂项 | Miscellaneous

\everymath{\displaystyle}
\setlength{\ULdepth}{0.75ex}
\newcommand \— {\,—\,}
\renewcommand \* {\;\!\!\!&\;\!\!\!}
\newcommand \nogap {\,\!}
\newcommand \Text [1] {\:\text{#1}\:}
\newcommand \serial [1] {\par\noindent{\bf #1}\ }
\newcommand \newconcept [1] {{\color{purple}\kaishu\underline{#1}}}
\newcommand \axiombasis [1] {$_{\bf\scriptscriptstyle #1}$}
\newcommand \sentence [1] {{\bf #1}}
\newcommand \existence {{\scriptsize 存在性}\,}
\newcommand \uniqueness {{\scriptsize 唯一性}\,}
\newcommand \limplyr {$\implies$\nogap}
\newcommand \rimplyl {$\impliedby$\nogap}
\newcommand \imply [2] {{\bf #1}\!$\implies$\!{\bf #2}}
\newcommand \biimply [2] {{\bf #1}\!$\iff$\!{\bf #2}}
\newcommand \method [1] {{\scriptsize(#1)}}

% 样式设置 | Format Settings

\ExplSyntaxOn
\FormatSetting {CatalogTitle} {
    \Large\HeavyFontC
    \begin{center}目录\quad\textbf{Content}\end{center}\bigskip
}
\FormatSetting {CatalogSection} {
    \BoldFontC
    \ifempty\subchaptername {
        \,{\scriptsize\chaptername}\ |\ \sectionname
    } {
        \,{\scriptsize\chaptername}\ \subchaptername\ |\ \sectionname
    }
    \vspace{-7.5pt}
}
\FormatSetting {CatalogIndex} {
    \BoldFontC{\Large\chaptername}\ \parameterA
    \vspace{-7.5pt}
}
\FormatSetting {IndexTitle} {
    \noindent
    \HeavyFontC{\huge\chaptername}\ \textbf{\Large\parameterA}
    \par
}
\FormatSetting {IndexSubchapter} {
    \BoldFontC
    \ifempty\subchaptername{
        {\Large\chaptername}
    } {
        {\normalsize\chaptername}\ {\Large\subchaptername}
    }
    \vspace{-5pt}
}
\FormatSetting {IndexSection} {
    \BoldFontC
    \ifempty\subchaptername {
        \,{\scriptsize\chaptername}\ |\ \sectionname
    } {
        \,{\scriptsize\chaptername}\ \subchaptername\ |\ \sectionname
    }
}
\FormatSetting {SectionHeading} {
    \noindent
    \bfseries\HeavyFontC
    \ifempty\subchaptername {
        {\Large\chaptername}{\huge\;|\;\sectionname}
    } {
        {\Large\chaptername}\ {\huge\subchaptername\;|\;\sectionname}
    }
    \par\bigskip
}
\FormatSetting {DefinitionName} {
    \vspace{4pt}\noindent
    \color{purple}\bfseries\ThinFontC\blockname
    \quad
}
\FormatSetting {DefinitionContent} {
    \color{purple}\kaishu\blockcontent
}
\FormatSetting {TheoremName} {
    \vspace{4pt}\noindent
    \color{teal}\bfseries\ThinFontC\blockname
    \quad
}
\FormatSetting {TheoremContent} {
    \color{teal}\kaishu\blockcontent
}
\FormatSetting {CorollaryName} {
    \vspace{4pt}\noindent
    \color{brown}\bfseries\ThinFontC\blockname
    \quad
}
\FormatSetting {CorollaryContent} {
    \color{brown}\kaishu\blockcontent
}
\FormatSetting {ProofName} {
    \par\noindent
    \kaishu\blockname
}
\FormatSetting {ProofContent} {
    \kaishu\blockcontent
    \par\smallskip
}
\FormatSetting {ConventionName} {
    \vspace{4pt}\noindent
    \color{darkgray}\bfseries\ThinFontC
    \ifequal\parameterB {sec} {
        \ifempty\subchaptername {
            \seclink{\chaptername}{\sectionlabel}
        } {
            \seclink{\chaptername}[\subchaptername]{\sectionlabel}
        }
    } {
        \ifequal\parameterB {cpt} {
            \cptlink{\chaptername}
        } {
            \ifequal\parameterB {subcpt} {
                \cptlink{\chaptername}[\subchaptername]
            } {
                \parameterB
            }
        }
    }
    中的设定
    \quad
}
\FormatSetting {ConventionContent} {
    \color{darkgray}\kaishu\blockcontent
}
\FormatSetting {StructureName} {
    \vspace{4pt}\noindent
    \color{violet}\bfseries\ThinFontC\blockname
    \quad
}
\FormatSetting {StructureContent} {
    \color{violet}\kaishu\blockcontent
}
\FormatSetting {ExampleName} {
    \vspace{4pt}\noindent
    \color{gray}\bfseries\ThinFontC\blockname
    \quad
}
\FormatSetting {ExampleContent} {
    \color{gray}\kaishu\blockcontent
}
\FormatSetting {BlockLink} {
    \bfseries\ThinFontC
    \ifempty\linktext {
        \,\blockname\,
    } {
        \,\linktext\,
    }
}
\FormatSetting {SectionLink} {
    \bfseries\BoldFontC
    \ifempty\linktext {
        \ifempty\subchaptername {
            \,{\footnotesize\chaptername}\,|\,\sectionname\,
        } {
            \,{\footnotesize\chaptername}\;\subchaptername\,|\,\sectionname\,
        }
    } {
        \,\linktext\,
    }
}
\FormatSetting {ChapterLink} {
    \bfseries\BoldFontC
    \ifempty\linktext {
        \ifempty\subchaptername {
            \,\chaptername\,
        } {
            \,{\footnotesize\chaptername}\;\subchaptername\,
        }
    } {
        \,\linktext\,
    }
}
\ExplSyntaxOff

% --------------------------------------------------
% | 正文区 | Body |
% --------------------------------------------------

\begin{document}

    % 封面 | Cover

    \pagestyle{empty}
    \begin{center}
	    \vspace*{\fill}
	    \SimpleSystem
	    \HeavyFontC\HeavyFontE\par\Huge Simple 线性代数
        $_{\!\scriptscriptstyle\mathbf{O}}$
	    \par\large Preview 20
	    \vspace*{\fill}
    \end{center}
    \newpage
    \pagestyle{plain}

    % 目录 | Catalog

    \MakeSimpleCatalog
    \TextCommand{\newpage}

    % 线性空间
    \CatalogPart{$\bm{\alpha_{1}}$}
    \ImportSection{线性空间}{定义}
    \ImportSection{线性空间}{基本性质}
    \ImportSection{线性空间}[子空间]{定义}
    \ImportSection{线性空间}[向量组]{张成空间}
    \ImportSection{线性空间}[向量组]{线性相关性}
    \ImportSection{线性空间}[向量组]{替换定理}
    \ImportSection{线性空间}{Hamel基}
    \ImportSection{线性空间}[子空间]{交&和}(交\&和)
    \ImportSection{线性空间}[子空间]{直和}
    %\ImportSection{线性空间}[子空间]{积空间}
    \ImportSection{线性空间}[子空间]{商空间}
    %\ImportSection{实线性空间}[复化]{定义}
    \ImportSection{有限维线性空间}{定义}
    \ImportSection{有限维线性空间}{基&维数}(基\&维数)
    \ImportSection{有限维线性空间}{维数公式}
    \CatalogCommand{\bigskip}

    % 线性映射
    \CatalogPart{$\bm{\alpha_{2}}$}
    \ImportSection{线性空间}[线性映射]{定义}
    \ImportSection{线性空间}[线性映射]{线性映射空间}
    \ImportSection{线性空间}[线性映射]{核&像}(核\&像)
    \ImportSection{线性空间}[线性映射]{复合}
    \ImportSection{线性空间}[线性映射]{逆}
    \ImportSection{有限维线性空间}{秩—零化度定理}(秩\—零化度定理)
    \CatalogCommand{\newpage}
    %\ImportSection{线性空间}[线性变换]{幂}
    %\ImportSection{线性空间}[线性变换]{多项式}
    %\ImportSection{线性空间}[线性泛函]{对偶空间}

    % 多重线性代数
    \CatalogPart{$\bm{\alpha_{3}}$}
    \ImportSection{线性空间}[多重线性型]{定义}
    \ImportSection{线性空间}[多重线性型]{对称型}
    \ImportSection{线性空间}[多重线性型]{交错型}
    \ImportSection{线性空间}[双线性型]{对称交错分解}
    %\ImportSection{实线性空间}[二次型]{定义}
    %\ImportSection{实线性空间}[二次型]{正定&负定}(正定\&负定)
    %\ImportSection{实线性空间}[二次型]{Sylvester惯性定理}
    \CatalogCommand{\bigskip}

    % 矩阵&行列式
    \CatalogPart{$\bm{\beta}$}
    \ImportSection{域}[矩阵]{定义}
    \ImportSection{域}[矩阵]{积}
    \ImportSection{域}[行列式]{定义}
    \ImportSection{域}[行列式]{基本性质}
    %\ImportSection{域}[矩阵]{逆}
    %\ImportSection{复数系}[矩阵]{复共轭}
    %\ImportSection{有限维线性空间}[矩阵表示]{向量&线性映射}(向量\&线性映射)
    %\ImportSection{有限维线性空间}[矩阵表示]{双线性型}
    %\ImportSection{有限维线性空间}[矩阵表示]{基变换}
    \CatalogCommand{\bigskip}

    % 特征值理论
    %\CatalogPart{$\bm{\gamma_{1}}$}
    %\ImportSection{线性空间}[线性变换]{特征值}
    %\ImportSection{线性空间}[线性变换]{广义特征空间}
    %\ImportSection{有限线性空间}[特征多项式]{定义}
    %\ImportSection{有限维线性空间}[特征多项式]{基本性质}
    %\ImportSection{有限维线性空间}[特征多项式]{代数重数}
    %\ImportSection{线性空间}[最小多项式]{定义}
    %\ImportSection{线性空间}[最小多项式]{基本性质}
    %\ImportSection{有限维线性空间}[矩阵表示]{对角化}
    %\CatalogCommand{\newpage}

    % 相似标准型
    \CatalogPart{$\bm{\gamma_{2}}$}
    \ImportSection{有限维线性空间}{广义特征空间分解}
    %\ImportSection{线性空间}[线性变换]{循环空间}
    \ImportSection{有限维线性空间}{幂零循环空间分解}
    \ImportSection{有限维线性空间}[矩阵表示]{Jordan标准型}
    %\ImportSection{有限维线性空间}[矩阵表示]{Frobenius标准型}
    \CatalogCommand{\bigskip}

    % 奇异值理论
    %\CatalogPart{$\bm{\gamma_{3}}$}
    %\ImportSection{有限维线性空间}[奇异值]{定义}
    %\ImportSection{有限维线性空间}[奇异值]{基本性质}
    %\ImportSection{有限维线性空间}[矩阵表示]{奇异值分解}

    % 度量结构
    \CatalogPart{$\bm{\delta}$}
    \ImportSection{赋范空间}{定义}
    %\ImportSection{赋范空间}[线性算子]{范数}
    %\ImportSection{赋范空间}[线性算子]{有界算子}
    \ImportSection{内积空间}{定义}
    \ImportSection{内积空间}{范数}
    \ImportSection{内积空间}{基本性质}
    \ImportSection{内积空间}{极化恒等式}
    \ImportSection{内积空间}[正交性]{定义}
    \ImportSection{内积空间}[正交性]{规范正交组}
    \ImportSection{内积空间}[正交性]{Gram—Schmidt过程}(Gram\—Schmidt过程)
    \ImportSection{内积空间}{极小化向量}
    \ImportSection{内积空间}[正交补]{定义}
    \ImportSection{内积空间}[正交补]{投影定理}
    %\ImportSection{内积空间}[正交补]{正交投影}
    \ImportSection{内积空间}{Riesz表示定理}
    \CatalogCommand{\bigskip}

    % 算子理论
    \CatalogPart{$\bm{\epsilon}$}
    \ImportSection{内积空间}[伴随映射]{定义}
    \ImportSection{内积空间}[伴随映射]{有限维性质}
    %\ImportSection{内积空间}{自伴算子}
    %\ImportSection{内积空间}{数值域}
    %\ImportSection{内积空间}{正规算子}
    %\ImportSection{内积空间}{等距映射}
    %\ImportSection{内积空间}{幺正算子}
    %\ImportSection{内积空间}{QR分解&Cholesky分解}(QR分解\&\gap Cholesky分解)
    \ImportSection{内积空间}{Moore—Penrose广义逆}(Moore\—Penrose广义逆)
    \CatalogCommand{\bigskip}

    % 索引 | Index

    \CatalogCommand{\bigskip}
    %\TextCommand{\newpage}
    %\MakeSimpleIndex{环}<Ring>
    \TextCommand{\newpage}
    \MakeSimpleIndex{域}<Field>
    \TextCommand{\newpage}
    \MakeSimpleIndex{线性空间}<Linear Space>
    \TextCommand{\newpage}
    \MakeSimpleIndex{有限维线性空间}<Finite Dimensional Linear Space>
    \TextCommand{\newpage}
    \MakeSimpleIndex{赋范空间}<Normed Space>
    \TextCommand{\newpage}
    \MakeSimpleIndex{内积空间}<Inner Product Space>
    %\TextCommand{\newpage}
    %\MakeSimpleIndex{复数系}<Complex Number System>
    %\TextCommand{\newpage}
    %\MakeSimpleIndex{实线性空间}<Real Linear Space>

\end{document}