
% 2025 Simple·System

% ./Section/线性空间/子空间/定义

\Convention{subcpt}{
    \serial{•}设 $U\subseteq V$ .
}

\par 对于线性空间 $V$ 的一些特定的子集 $U$ , 我们可以直接在 $U$ 上赋予与 $V$ 一致的代数结构(域和运算)而使之成为线性空间.

\Definition{子空间}{
    $\braI{U\and\bfF\and+\and\cdot}$ 是 $\braI{V\and\bfF\and+\and\cdot}$ 的\newconcept{子空间}, 当且仅当 $\braI{U\and\bfF\and+\and\cdot}$ 是线性空间.
}

\par $V$ 的最小子空间和最大子空间分别是 $\braIII{\bm 0}$ 和 $V$ , 我们将除 $V$ 自身外的子空间统称为 $V$ 的\newconcept{真子空间}.

\Theorem{子空间的判定准则}{
    $\braI{U\and\bfF\and+\and\cdot}$ 是 $\braI{V\and\bfF\and+\and\cdot}$ 的子空间, 当且仅当其满足如下性质:
    \serial{•}$U\ne\emptyset$ ;
    \serial{•}$\lambda\bm u+\mu\bm v\in U$ 对一切 $\lambda\and\mu\in\bfF$ 和 $\bm u\and\bm v\in U$ 成立.
}
\Proof{$\implies$\nogap 证明}{
    \par 这是显然的.\ProofEndT
}
\Proof{$\impliedby$\nogap 证明}{
    \par $a\bm u+b\bm v\in U$ 保证了 $U$ 对加法和数乘运算封闭. 由 $U$ 非空知 $\bm 0=0\bm u\in U$ 成立, 其中 $\bm u\in U$ ; 因为数乘运算与 $V$ 相同, 因此 $U$ 有和 $V$ 相同的乘法幺元; 同时对一切 $\bm u\in U$ 有 $-\bm u=\braI{-1}\bm u\in U$ ; 其余\link{线性空间}{线性空间}要求的运算律显然成立, 因为它们在包含 $U$ 的线性空间 $V$ 上成立.\ProofEndT
}

\par 该准则中对 $U$ 非空的要求是必须的, 因为空集满足两条封闭性条件但不是线性空间.

\Convention{\cptlink{线性空间}[子空间]中的设定}{
    \serial{•}设 $U\and V_{1}\and V_{2}\and\cdots\and V_{n}$ 是 $V$ 的子空间.
}

\newpage
