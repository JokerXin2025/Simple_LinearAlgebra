
% 2025 Z·MathLib

% Article/线性空间/广义特征空间/有限维刻画

\Proof{证明}{

  先证 $\ker(T-\lambda I)^\nu \subseteq E_T^*(\lambda)$:对任意 $v \in \ker(T-\lambda I)^\nu$,由定义 $(T-\lambda I)^\nu v = 0$,故 $v \in E_T^*(\lambda)$。

  再证 $E_T^*(\lambda) \subseteq \ker(T-\lambda I)^\nu$:设 $v \in E_T^*(\lambda)$,则存在正整数 $k$ 使得 $(T-\lambda I)^k v = 0$。考虑多项式 $f(x) = (x-\lambda)^k$ 和 $m(x) = (x-\lambda)^\nu q(x)$。由于 $q(\lambda) \neq 0$,$q(x)$ 与 $(x-\lambda)^k$ 互质,故存在多项式 $a(x), b(x)$ 使得:
  \[
  a(x)q(x) + b(x)(x-\lambda)^k = 1.
  \]
  代入 $x = T$ 得:
  \[
  a(T)q(T) + b(T)(T-\lambda I)^k = I.
  \]
  于是:
  \[
  v = a(T)q(T)v + b(T)(T-\lambda I)^k v = a(T)q(T)v.
  \]
  现在计算:
  \[
  (T-\lambda I)^\nu v = (T-\lambda I)^\nu a(T)q(T)v = a(T)q(T)(T-\lambda I)^\nu v.
  \]
  但由 $m(T) = 0$ 得 $(T-\lambda I)^\nu q(T) = 0$,故 $(T-\lambda I)^\nu q(T)v = 0$,从而 $(T-\lambda I)^\nu v = 0$。因此 $v \in \ker(T-\lambda I)^\nu$。

  综上,$E_T^*(\lambda) = \ker(T-\lambda I)^\nu$。
}

\newpage
