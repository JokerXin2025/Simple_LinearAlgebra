
% 2025 Simple·System

% ./Section/线性空间/最小多项式/基本性质

\par 根据定义, 我们知道\link{线性空间}[线性变换]{幂零变换}的最小多项式都形如 $\mu\braI{x}=x^{n}$ , 其中 $n\in\bfN$ , 下面的定理将进一步表明 $0$ 是幂零变换唯一的特征值.

\Theorem{最小多项式的零点等价于特征值}(最小多项式的零点\nogap $\iff$\nogap 特征值){
    设 $\lambda\in\bfF$ , 则 $\lambda$ 是 $T$ 的特征值当且仅当 $\mu\braI{\lambda}=0$ .
}
\Proof{\limplyr 证明}{
    \par $\bm v_{0}\in V$ 是对应于特征值 $\lambda$ 的特征向量, 则
    \begin{align*}
        \Forall{k}{\bfN}
        \braI{T-\lambda I}\braI{T^{k}\bm v_{0}}=T^{k+1}\bm v_{0}-\lambda T^{k}\bm v_{0}=T^{k}\braI{T\bm v_{0}-\lambda\bm v_{0}}=T^{k}\bm 0=\bm 0\ ,
    \end{align*}
    从而 $\im pT\subseteq\ker\braI{T-\lambda I}$ . 假设有 $p\lambda\ne0$ , 则由 $\deg p>0$ 可知 $\deg q<\deg p$ , 其中 $q:x\mapsto\sfrac{px}{x-\lambda}$ . 这说明 $q$ 是 $T$ 的零化多项式, 这与 $p$ 的最小性矛盾. 因此 $p\lambda=0$ , 这就完成了证明.\ProofEndT
}
\Proof{\rimplyl 证明}{
    \par 设 $p$ 是 $T$ 的最小多项式, $\lambda\in\bfF$ 是 $p$ 的零点, 则对任意 $\bm v\in V$ 有
    \begin{align*}
        \braI{T-\lambda I}\braI{qT}\bm v=\braI{qT}\braI{T-\lambda I}\bm v=\braI{qT}\braI{T\bm v-\lambda\bm v}=\braI{qT}T\bm v-\lambda\cdot\braI{qT}\bm v=\bm 0-\lambda\cdot\bm 0=\bm 0\ ,
    \end{align*}
    其中 $q:x\mapsto\sfrac{p(x)}{x-\lambda}$ . 这说明 $\im qT\subseteq\ker\braI{T-\lambda I}$ , 因而 $\dim\ker\braI{T-\lambda I}\geqslant\dim\im qT=n-\deg q=n-\braI{\deg p-1}=1$ , 这就表明 $\lambda$ 是 $T$ 的特征值.\ProofEndT
}

\par 对于 $T$ 的特征值 $\lambda$ , 我们定义 $\lambda$ 的\newconcept{指数}为其在最小多项式 $\mu$ 中作为零点的重数.

\Theorem{广义特征空间的终止点}{
    设 $T$ 有最小多项式, $\lambda$ 是 $T$ 的特征值且指数为 $\nu$ , 则
    \begin{align*}
        \ker\braI{T-\lambda\bm I}^{\nu-1}\subset\ker\braI{T-\lambda\bm I}^{\nu}=E_{T}^{*}\braI{\lambda}\ .
    \end{align*}
}
\Proof{证明}{
    \par 我们记
    \begin{align*}
        \mu_{0}\braI{x}=\frac{\mu\braI{x}}{\braI{x-\lambda}^{\nu}}\ ,
    \end{align*}
    由 $\mu$ 和 $\nu$ 的定义可知 $\mu T=\bm 0$ 且 $\mu_{0}\braI{\lambda}\ne 0$ .

\newpage

    \par 先证明 $\ker\braI{T-\lambda\bm I}^{\nu-1}\subset\ker\braI{T-\lambda\bm I}^{\nu}$ . 假设该式不成立, 那么
    \begin{align*}
        \ker\braI{T-\lambda\bm I}^{\nu-1}=\ker\braI{T-\lambda\bm I}^{\nu}=E_{T}^{*}\braI{\lambda}\ ,
    \end{align*}
    而由 $\mu T=\braI{T-\lambda\bm I}^{\nu}\mu_{0}T$ 知 $\mu_{0}T\braI{V}\subseteq E_{T}^{*}\braI{\lambda}$ , 令 $\mu'\braI{x}=\braI{x-\lambda}^{\nu-1}\mu_{0}\braI{x}$ , 则
    \begin{align*}
        \mu'T=\braI{T-\lambda\bm I}^{\nu-1}\mu_{0}T=\bm 0\ ,
    \end{align*}
    这与 $\mu$ 的最小性矛盾.
    \par 再证明 $\ker\braI{T-\lambda\bm I}^{\nu}=\ker\braI{T-\lambda\bm I}^{\nu+1}$ . 假设存在某个非零向量
    \begin{align*}
        \bm v\in\ker\braI{T-\lambda\bm I}^{\nu+1}\setminus\ker\braI{T-\lambda\bm I}^{\nu}\ ,
    \end{align*}
    设 $\bm v'=\braI{T-\lambda\bm I}^{\nu}\bm v$ , 则 $\bm v'$ 是对应于 $\lambda$ 的特征向量, 此时
    \begin{align*}
        \braI{\mu T}\bm v=\braI{\mu_{0}T}\bm v'=\mu_{0}\braI{\lambda}\bm v'\ne\bm 0\ ,
    \end{align*}
    这就导出了矛盾.\ProofEndT
}

\newpage
