
% 2025 Simple·System

% ./Section/内积空间/范数

\par 我们要求任何内积都满足正定性, 这确保每一种内积运算都能诱导出一种范数.

\Definition{范数}{
    设 $V$ 是一个内积空间, 则对任意 $\bm v\in V$ , 定义 $\bm v$ 的\newconcept{范数}为
    \begin{align*}
        \norm{\bm v}:=\sqrt{\braV{\bm v\and\bm v}}\ .
    \end{align*}
}

\vspace{-4pt}

\Corollary{范数的基本性质}{
    设 $V$ 是一个内积空间, 则对任意的 $\bm v\in V$ 有
    \par\noindent 1. 正定性: $$\Forall{\bm v}{V}\norm{\bm v}\geqslant0\ ,$$ 其中等号当且仅当 $\bm v=\bm 0$ 时成立.
    \par\noindent 2. 线性: $$\ForallII{\bm v}{V}{\lambda}{\bfF}\norm{\lambda\bm v}=\braIV{\lambda}\norm{\bm v}\ .$$
}

\vspace{-4pt}

\Theorem{余弦定理}{
    设 $V$ 是一个内积空间, 则
    \begin{align*}
        \Forall{\bm v_{1}\and\bm v_{2}}{V}
        \norm{\bm v_{1}+\bm v_{2}}^{2}=\norm{\bm v_{1}}^{2}+\norm{\bm v_{2}}^{2}+2\Re\braV{\bm v_{1}\and\bm v_{2}}\ .
    \end{align*}
}
\Proof{证明}{
    \begin{align*}
        \norm{\bm v_{1}+\bm v_{2}}^{2}&=\braV{\bm v_{1}+\bm v_{2}\and\bm v_{1}+\bm v_{2}}\\[-2mm]
        &=\braV{\bm v_{1}\and\bm v_{1}}+\braV{\bm v_{2}\and\bm v_{2}}+\braV{\bm v_{1}\and\bm v_{2}}+\overline{\braV{\bm v_{1}\and\bm v_{2}}}\\[-2mm]
        &=\norm{\bm v_{1}}^{2}+\norm{\bm v_{2}}^{2}+2\Re\braV{\bm v_{1}\and\bm v_{2}}\ .\ProofEndF
    \end{align*}
}

\Theorem{平行四边形恒等式}{
    设 $V$ 是一个内积空间, 则
    \begin{align*}
        \Forall{\bm v_{1}\and\bm v_{2}}{V}
        \norm{\bm v_{1}+\bm v_{2}}^{2}+\norm{\bm v_{1}-\bm v_{2}}^{2}=2\braI{\norm{\bm v_{1}}^{2}+\norm{\bm v_{2}}^{2}}\ .
    \end{align*}
}
\Proof{证明}{
    \begin{align*}
        \
    \end{align*}
}

\Convention{cpt}{
    \serial{•}对任意内积空间 $V$ 和任意 $\bm v\in V$ : 令 $\norm{\bm v}$ 表示 $\bm v$ 的范数.
}

\newpage
