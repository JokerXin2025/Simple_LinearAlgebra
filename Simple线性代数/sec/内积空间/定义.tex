
% 2025 Simple·System

% ./Section/内积空间/定义

\Definition{内积空间}{
    设 $\braI{V\and\bfF\and+\and\cdot}$ 是一个实线性空间或复线性空间, 则 $\braI{\!\braI{V\and\bfF\and+\and\cdot}\and\braV{\cdot\and\cdot}\!}$ 是\newconcept{内积空间}, 当且仅当其内积 $\braV{\cdot\and\cdot}:V^{2}\to\bfF$ 满足如下性质:
    \serial{1.}正定性: $$\Forall{\bm v}{V}\braV{\bm v\and\bm v}\geqslant 0\ ,$$ 其中等号当且仅当 $\bm v=\bm 0$ 时成立;
    \serial{2.}共轭对称性: $$\Forall{\bm u\and\bm v}{V}\braV{\bm u\and\bm v}=\overline{\braV{\bm v\and\bm u}}\ ;$$
    \serial{3.}第一元线性: $$\braV{\bm u+\bm v\and\bm w}=\braV{\bm u\and\bm w}+\braV{\bm v\and\bm w}\AND\braV{\lambda\bm u\and\bm v}=\lambda\braV{\bm u\and\bm v}$$ 对一切 $\bm u\and\bm v\and\bm w\in V$ 和 $\lambda\in\bfF$ 成立.
}

\par 当 $V$ 为实线性空间时, 对应的内积空间称为\gap\newconcept{Euclid空间}, 此时共轭对称性退化为对称性 $\braV{\bm v_{1}\and\bm v_{2}}=\braV{\bm v_{2}\and\bm v_{1}}$ ; 当 $V$ 为复线性空间时, 对应的内积空间称为\newconcept{酉空间}. 我们不会对建立在 $\bfR$ 和 $\bfC$ 之外的域上的线性空间讨论内积, 因此在\cptlink{内积空间}中将默认 $\bfF\in\braIII{\bfR\and\bfC}$ .

\Corollary{第二元共轭线性}{
    设 $V$ 是一个内积空间, 则
    \begin{align*}
        \braV{\bm u\and\bm v+\bm w}=\braV{\bm u\and\bm v}+\braV{\bm u\and\bm w}\AND\braV{\bm u\and\lambda\bm v}=\overline{\lambda}\braV{\bm u\and\bm v}
    \end{align*}
    对一切 $\bm u\and\bm v\and\bm w\in V$ 和 $\lambda\in\bfF$ 成立.
}

\vspace{-4pt}

\Convention{cpt}{
    \serial{•}设 $U\and V$ 是内积空间.
    \serial{•}对任意线性空间 $U\and V$ : 令 $\mathcal{L}\braI{U\and V}$ 表示从 $U$ 到 $V$ 的线性映射空间.
    \serial{•}不产生歧义时, 令 $\bm O$ 和 $\bm I$ 表示任何零矩阵和单位矩阵.
    \serial{•}对任意内积空间 $V$ : 令 $\bm I_{V}$ 表示 $V$ 上的恒等映射.
    \serial{•}不产生歧义时, 用 $\bfR^{n}$ 代表内积空间 $\braI{\!\braI{\bfR^{n}\and\bfR\and+\and\cdot}\and\cdot}$ , 其中 $\cdot$ 表示 $\bfR^{n}$ 上的点积.
}

\newpage
