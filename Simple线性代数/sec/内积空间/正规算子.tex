
% 2025 Simple·System

% ./Section/内积空间/正规算子

\Definition{正规算子}{
    称 $T\in\mathcal{L}\braI{V}$ 是正规算子, 当且仅当存在 $T$ 的伴随映射且 $T$ 与 $T^{*}$ 可交换.
}

\vspace{-4pt}

\Corollary{自伴算子是正规算子}(自伴算子\nogap $\implies$\nogap 正规算子){
    任何自伴算子都是正规算子.
}

\vspace{-4pt}

\Corollary{正规算子的矩阵}{
    任何正规算子在规范正交基下的矩阵都是正规矩阵.
}
\vspace{-4pt}

\Theorem{正规算子的等价条件}{
    设 $T\in\mathcal{L}\braI{V}$ , 下面的各命题等价:
    \serial{1.}$T$ 是正规算子.
    \serial{2.}$\norm{T\bm v}=\norm{T^{*}\bm v}$ 对一切 $\bm v\in V$ 成立.
    \serial{3.}存在可交换的自伴算子 $A\and B\in\mathcal{L}\braI{V}$ 使得 $T=A+\rmi B$ .
}
\Proof{\imply{1}{2}证明}{
    \par 对任意 $\bm v\in V$ 有
    \begin{align*}
        \braV{T\bm v\and T\bm v}=\braV{\bm v\and T^{*}T\bm v}=\braV{\bm v\and TT^{*}\bm v}=\braV{T^{*}\bm v\and T^{*}\bm v}\ ,
    \end{align*}
    于是 $\norm{T\bm v}=\norm{T^{*}\bm v}$ , 其中最后一个等号使用了伴随映射的对合律.\ProofEndT
}
\Proof{\imply{2}{1}证明}{
    \par 对任意 $\bm v\in V$ 有
    \begin{align*}
        \braV{\!\braI{T^{*}T-TT^{*}}\bm v\and\bm v}=\braV{T^{*}T\bm v\and\bm v}-\braV{TT^{*}\bm v\and\bm v}=\braV{T\bm v\and T\bm v}-\braV{T^{*}\bm v\and T^{*}\bm v}=0\ ,
    \end{align*}
    于是 $W\braI{T^{*}T-TT^{*}}=\braIII{0}$ , 同时注意到
    \begin{align*}
        \Forall{\bm v_{1}\and\bm v_{2}}{V}
        \braV{\!\braI{T^{*}T-TT^{*}}\bm v_{1}\and\bm v_{2}}&=\braV{T^{*}T\bm v_{1}\and\bm v_{2}}-\braV{TT^{*}\bm v_{1}\and\bm v_{2}}\\[-2mm]
        &=\braV{T\bm v_{1}\and T\bm v_{2}}-\braV{T^{*}\bm v_{1}\and T^{*}\bm v_{2}}\\[-2mm]
        &=\braV{\bm v_{1}\and T^{*}T\bm v_{2}}-\braV{\bm v_{1}\and TT^{*}\bm v_{2}}\\[-2mm]
        &=\braV{\bm v_{1}\and\braI{T^{*}T-TT^{*}}\bm v_{2}}\ ,
    \end{align*}
    因此 $T^{*}T-TT^{*}$ 是自伴算子, 由\link{内积空间}{数值域零定理}即得 $T^{*}T=TT^{*}$ .\ProofEndT
}
\Proof{\imply{1}{3}证明}{
    \par 令
    \begin{align*}
        A=\frac{T+T^{*}}{2}\AND B=\frac{T-T^{*}}{2\rmi}\ ,
    \end{align*}
    显然 $A$ 和 $B$ 是自伴算子并且 $T=A+\rmi B$ , 由
    \begin{align*}
        AB-BA=\frac{T^{*}T-TT^{*}}{2\rmi}=\bm 0
    \end{align*}
    知 $A$ 和 $B$ 可交换, 这就完成了证明.\ProofEndT
}
\Proof{\imply{3}{1}证明}{
    \par 由 $A$ 和 $B$ 是自伴算子可知 $T^{*}=A-\rmi B$ , 于是有
    \begin{align*}
        T^{*}T-TT^{*}=2\rmi\braI{AB-BA}=\bm 0\ ,
    \end{align*}
    从而 $T$ 和 $T^{*}$ 可交换.\ProofEndT
}

\Theorem{正规算子的核&像}(正规算子的核\&像){
    设 $T\in\mathcal{L}\braI{V}$ 是正规算子, 则
    \serial{1.}$\ker T=\ker T^{*}$ .
    \serial{2.}当 $V$ 是有限维空间时有 $\im T=\im T^{*}$ , 并且 $V=\ker T\oplus\im T$ .
}
\Proof{\sentence{1}证明}{
    \par 对任意 $\bm v\in V$ , 我们有
    \begin{align*}
        T\bm v=\bm 0\iff\norm{T\bm v}=0\iff\norm{T^{*}\bm v}=0\iff T^{*}\bm v=\bm 0\ ,
    \end{align*}
    这就证明了 $\ker T=\ker T^{*}$ .\ProofEndT
}
\Proof{\sentence{2}证明}{
    \par 对\sentence{1}的结果应用\link{内积空间}[伴随映射]{有限维伴随映射的像}(有限维伴随映射的性质)可得
    \begin{align*}
        \im T=\braI{\ker T^{*}}^{\perp}=\braI{\ker T}^{\perp}=\im T^{*}\ ,
    \end{align*}
    并且
    \begin{align*}
        V=\ker T\oplus\braI{\ker T}^{\perp}=\ker T\oplus\im T^{*}=\ker T\oplus\im T\ .\ProofEndF
    \end{align*}
}

\Theorem{正规算子的特征向量}{
    设 $T\in\mathcal{L}\braI{V}$ 是正规算子, 则
    \serial{1.}对任意 $\bm v\in V$ 和 $\lambda\in\bfF$ , $T\bm v=\lambda\bm v$ 当且仅当 $T^{*}\bm v=\overline{\lambda}\bm v$ .
    \serial{2.}$T$ 的对应于不同特征值的特征向量正交.
}
\Proof{\sentence{1}证明}{
    \begin{align*}
        \Forall{\lambda}{\bfF}
        \braI{T-\lambda\bm I}\braI{T-\lambda\bm I}^{*}&=\braI{T-\lambda\bm I}\braI{T^{*}-\overline{\lambda}\bm I}\\[-2mm]
        &=TT^{*}-\lambda T^{*}-\overline{\lambda}T+\braIV{\lambda}^{2}\bm I\\[-2mm]
        &=T^{*}T-\overline{\lambda}T-\lambda T^{*}+\braIV{\lambda}^{2}\bm I\\[-2mm]
        &=\braI{T^{*}-\overline{\lambda}\bm I}\braI{T-\lambda\bm I}\\[-2mm]
        &=\braI{T-\lambda\bm I}^{*}\braI{T-\lambda\bm I}\ ,
    \end{align*}
    即 $T-\lambda\bm I$ 是正规算子, 于是
    \begin{align*}
        \norm{\braI{T-\lambda\bm I}\bm v}=\norm{\braI{T-\lambda\bm I}^{*}\bm v}=\norm{\braI{T^{*}-\overline{\lambda}\bm I}\bm v}\ ,
    \end{align*}
    因此 $\braI{T-\lambda\bm I}\bm v=\bm 0$ 当且仅当 $\braI{T^{*}-\overline{\lambda}\bm I}\bm v=\bm 0$ , 也即 $T\bm v=\lambda\bm v$ 当且仅当 $T^{*}\bm v=\overline{\lambda}\bm v$ .\ProofEndT
}
\Proof{\sentence{2}证明}{
    \par 设 $\bm v_{1}$ 和 $\bm v_{2}$ 是分别对应于不同特征值 $\lambda_{1}$ 和 $\lambda_{2}$ 的特征向量, 那么
    \begin{align*}
        \braI{\lambda_{1}-\lambda_{2}}\braV{\bm v_{1}\and\bm v_{2}}=\braV{\lambda_{1}\bm v_{1}\and\bm v_{2}}-\braV{\bm v_{1}\and\overline{\lambda_{2}}\bm v_{2}}=\braV{T\bm v_{1}\and\bm v_{2}}-\braV{\bm v_{1}\and T^{*}\bm v_{2}}=0\ ,
    \end{align*}
    由于 $\lambda_{1}\ne\lambda_{2}$ , 故上式表明 $\bm v_{1}\perp\bm v_{2}$ .\ProofEndT
}

\Theorem{复谱定理}{
    设 $V$ 是有限维酉空间且 $T\in\mathcal{L}\braI{V}$ , 则 $T$ 是正规算子当且仅当 $T$ 在某个规范正交基下的矩阵是对角矩阵.
}
\Proof{$\implies$证明\method{强归纳法}}{
    \par 由于 $T$ 的最小多项式分裂, 我们可以对 $T$ 的某个Jordan标准型应用\link{内积空间}[正交性]{上三角矩阵的规范正交化}以得到 $T$ 的某个上三角矩阵
    \begin{align*}
        \bm A'=\matrix{ccccc}{
            A'_{1,1} \* \cdots \* A'_{1,j} \* \cdots \* A'_{1,n} \\[-1mm]
            \vdots \* \ddots \* \* \rdots \* \vdots \\[-1mm]
            0 \* \* A'_{i,j} \* \* A'_{i,n} \\[-1mm]
            \vdots \* \rdots \* \* \ddots \* \vdots \\[-1mm]
            0 \* \cdots \* 0 \* \cdots \* A'_{n,n}
        }\ ,
    \end{align*}
    又因为 $T$ 是正规算子, 所以对任意正整数 $m\leqslant n$ 有 $\norm{T\bm e_{m}}=\norm{T^{*}\bm e_{m}}$ 成立, 又由\link{内积空间}[正交性]{规范正交性引理}知
    \begin{align*}
        \norm{T\bm e_{m}}^{2}=\sum_{k=1}^{m}\braIV{A'_{k,m}}^{2}\AND\norm{T^{*}\bm e_{m}}^{2}=\sum_{k=m}^{n}\braIV{A'_{m,k}}^{2}\ ,
    \end{align*}
    于是
    \begin{align*}
        \sum_{k=1}^{m}\braIV{A'_{k,m}}^{2}=\sum_{k=m}^{n}\braIV{A'_{m,k}}^{2}\ ,
    \end{align*}
    下面我们将证明对一切不同的 $p\and q\leqslant n$ 都有 $A'_{p,q}=0$ 成立, 从而 $\bm A'$ 是对角矩阵.
    \par 对正整数 $p$ 进行归纳. 当 $p=1$ 时, 在上面的等式中令 $m=1$ 即有
    \begin{align*}
        \sum_{k=2}^{n}\braIV{A'_{1,k}}^{2}=\sum_{k=1}^{n}\braIV{A'_{1,k}}^{2}-\sum_{k=1}^{1}\braIV{A'_{k,1}}^{2}=0\ .
    \end{align*}
    \par 假设待证结论在 $p<r$ 时成立, 那么在上面的等式中令 $m=r$ 即有
    \begin{align*}
        \sum_{k=r+1}^{n}\braIV{A'_{r,k}}^{2}=\sum_{k=r}^{n}\braIV{A'_{r,k}}^{2}-\sum_{k=1}^{r}\braIV{A'_{k,r}}^{2}+\underbrace{\sum_{k=1}^{r-1}\braIV{A'_{k,r}}^{2}}_{0}=0\ ,
    \end{align*}
    于是待证结论在 $p=r$ 时成立, 进而其对一切正整数 $p\leqslant n$ 成立.\ProofEndT
}
\Proof{$\impliedby$证明}{
    \par 记 $\bm A$ 为 $T$ 在某个规范正交基下的矩阵, 则作为\link{内积空间}[伴随映射]{伴随映射的矩阵}, $T^{*}$ 在该基下的矩阵为 $\overline{\bm A}\,^{\rmT}$ , 若 $\bm A$ 是对角矩阵, 则显然有 $\overline{\bm A}\,^{\rmT}\bm A=\bm A\overline{\bm A}\,^{\rmT}$ , 于是 $T^{*}T=TT^{*}$ .\ProofEndT
}

\newpage
