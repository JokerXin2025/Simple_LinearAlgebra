
% 2025 Simple·System

% ./Section/内积空间/等距映射

\Convention{sec}{
    \serial{•} 设 $T\in\mathcal{L}\braI{U\and V}$ .
}

\vspace{-4pt}

\Definition{等距映射}{
    线性映射 $T$ 是\newconcept{等距映射}, 当且仅当其满足:
    \serial{•}结构保持性:
    \begin{align*}
        \braV{\bm u_{1}\and\bm u_{2}}=\braV{T\bm u_{1}\and T\bm u_{2}}
    \end{align*}
    对一切 $\bm u_{1}\and\bm u_{2}\in U$ 成立.
}

\par 下面我们将证明等距映射使得两个内积空间具有高度关联性, 因此我们又将作为等距映射的双射称为内积空间之间的\newconcept{同构映射}, 并且如果两个内积空间之间存在同构映射, 则称它们是\newconcept{同构}的.

\Theorem{等距映射的等价条件}{
    下列各命题等价:
    \serial{1.}$T$ 是等距映射.
    \serial{2.}$\norm{\bm u}=\norm{T\bm u}$ 对一切 $\bm u\in U$ 成立.
    \serial{3.}$T^{*}T=\bm I$ .
    \serial{4.}设 $(\bm e_{k})_{k\in I}$ 是 $U$ 中的规范正交组, 则 $(T\bm e_{k})_{k\in I}$ 规范正交.
}
\Proof{\imply{1}{2}证明}{
    \par 令 $\braI{\bm u_{1}\and\bm u_{2}}=\braI{\bm u\and\bm u}$ 即证.\ProofEndT
}
\Proof{\imply{1}{3}证明}{
    \par 证明.
}
\Proof{\imply{1}{4}证明}{
    \begin{align*}
        \Forall{p\and q}{I}
        \braV{T\bm e_{p}\and T\bm e_{q}}=\braV{\bm e_{p}\and\bm e_{q}}=\delta_{p,q}\ .\ProofEndF
    \end{align*}
}
\Proof{\imply{2}{1}证明}{
    \par 由极化恒等式有
    \begin{small}\begin{align*}
        \braV{T\bm u_{1}\and T\bm u_{2}}&=\frac{\norm{T\bm u_{1}+T\bm u_{2}}^{2}-\norm{T\bm u_{1}-T\bm u_{2}}^{2}+\norm{T\bm u_{1}+\rmi T\bm u_{2}}^{2}\rmi-\norm{T\bm u_{1}-\rmi T\bm u_{2}}^{2}\rmi}{4}\\[1mm]
        &=\frac{\norm{T\braI{\bm u_{1}+\bm u_{2}}}^{2}-\norm{T\braI{\bm u_{1}-\bm u_{2}}}^{2}+\norm{T\braI{\bm u_{1}+\rmi\bm u_{2}}}^{2}\rmi-\norm{T\braI{\bm u_{1}-\rmi\bm u_{2}}}^{2}\rmi}{4}\\[1mm]
        &=\frac{\norm{\bm u_{1}+\bm u_{2}}^{2}-\norm{\bm u_{1}-\bm u_{2}}^{2}+\norm{\bm u_{1}+\rmi\bm u_{2}}^{2}\rmi-\norm{\bm u_{1}-\rmi\bm u_{2}}^{2}\rmi}{4}\\[-1mm]
        &=\braV{\bm u_{1}\and\bm u_{2}}\ .\ProofEndF\\[-15mm]
    \end{align*}\end{small}
}
\Proof{\imply{2}{3}证明}{
    \par 对于任意 $\bm u\in U$ , 我们有
    \begin{align*}
        \braV{\!\braI{\bm I_{U}-T^{*}T}\bm u\and\bm u}&=\braV{\bm u\and\bm u}-\braV{T^{*}T\bm u\and\bm u}\\[-2mm]
        &=\braV{\bm u\and\bm u}-\braV{T\bm u\and T\bm u}\\[-2mm]
        &=\norm{\bm u}^{2}-\norm{T\bm u}^{2}\\[-2mm]
        &=0\ ,
    \end{align*}
    又因为 $\bm I_{U}-T^{*}T$ 是自伴算子, 于是 $T^{*}T=\bm I_{U}$ .
}
\Proof{\imply{3}{1}证明}{
    \begin{align*}
        \braV{T\bm u\and T\bm v}=\braV{\bm u\and T^{*}T\bm v}=\braV{\bm u\and\bm v}\ .\ProofEndF
    \end{align*}
}
\Proof{\imply{4}{1}证明}{
    \par 对任意 $\bm u_{1}\and\bm u_{2}\in U$ , 设
    \begin{align*}
        \bm u_{1}=\sum_{k=1}^{n}a_{k}\bm e_{k}\AND\bm u_{2}=\sum_{k=1}^{n}b_{k}\bm e_{k}\ ,
    \end{align*}
    则
    \begin{align*}
        \braV{\bm u_{1}\and\bm u_{2}}=\braV{\sum_{k=1}^{n}a_{k}\bm e_{k}\and\sum_{k=1}^{n}b_{k}\bm e_{k}}=\sum_{i=1}^{n}\sum_{j=1}^{n}a_{i}\overline{b_{j}}\braV{\bm e_{i}\and\bm e_{j}}\ ,\\[-14mm]
    \end{align*}
    \begin{align*}
        \braV{T\bm u_{1}\and T\bm u_{2}}=\braV{\sum_{k=1}^{n}a_{k}T\bm e_{k}\and\sum_{k=1}^{n}b_{k}T\bm e_{k}}=\sum_{i=1}^{n}\sum_{j=1}^{n}a_{i}\overline{b_{j}}\braV{T\bm e_{i}\and T\bm e_{j}}\ ,
    \end{align*}
    由 $(\bm e_{k})_{k=1}^{n}$ 和 $(T\bm e_{k})_{k=1}^{n}$ 都是规范正交组可知
    \begin{align*}
        \braV{\bm e_{i}\and\bm e_{j}}=\braV{T\bm e_{i}\and T\bm e_{j}}=\delta_{i,j}\ ,
    \end{align*}
    对一切正整数 $i\and j\leqslant n$ 成立, 于是 $\braV{\bm u_{1}\and\bm u_{2}}=\braV{T\bm u_{1}\and T\bm u_{2}}$ .\ProofEndT
}
\Proof{\imply{4}{3}证明}{
    \par 证明.
}

\Theorem{等距映射的矩阵}{
    设 $U$ 和 $V$ 分别是以 $B_{U}$ 和 $B_{V}$ 为规范正交基的非零有限维空间, 记
    \begin{align*}
        \bm A=\matrix{ccc}{
            A_{11} \* \cdots \* A_{1n} \\[-1mm]
            \vdots \* \ddots \* \vdots \\[-1mm]
            A_{m1} \* \cdots \* A_{mn}
        }
    \end{align*}
    是等距映射 $T\in\mathcal{L}\braI{U\and V}$ 在 $B_{U}$ 和 $B_{V}$ 下的矩阵, 则 $\bfF^{m}$ 中的向量组
    \begin{align*}
        \braI{(A_{k1})_{k=1}^{m}\and\cdots\and(A_{kn})_{k=1}^{m}}
    \end{align*}
    是规范正交的.
}
\Proof{证明}{
    \par 
}

\newpage
