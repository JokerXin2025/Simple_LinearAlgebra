
% 2025 Simple·System

% ./Section/域/行列式/基本性质

% 设AB为同阶方阵
% 设B是V的基,M(v)表示v在B下的矩阵表示

\Corollary{行列式的规范性}{
    $\det\bm I=1$ .
}

\vspace{-4pt}

\Theorem{行列式的交错多重线性}{
    \serial{1.}多重可加性:
    \begin{align*}
        \determinant{ccc}{
            \* \vdots \* \\[-1mm]
            a_{1}+b_{1} \* \cdots \* a_{n}+b_{n} \\[-1mm]
            \* \vdots \*
        }=\determinant{ccc}{
            \* \vdots \* \\[-1mm]
            a_{1} \* \cdots \* a_{n} \\[-1mm]
            \* \vdots \*
        }+\determinant{ccc}{
            \* \vdots \* \\[-1mm]
            b_{1} \* \cdots \* b_{n} \\[-1mm]
            \* \vdots \*
        }\ ,\\[-14mm]
    \end{align*}
    \begin{align*}
        \determinant{ccc}{
            \* a_{1}+b_{1} \* \\[-1mm]
            \cdots \* \vdots \* \cdots \\[-1mm]
            \* a_{n}+b_{n} \*
        }=\determinant{ccc}{
            \* a_{1} \* \\[-1mm]
            \cdots \* \vdots \* \cdots \\[-1mm]
            \* a_{n} \*
        }+\determinant{ccc}{
            \* b_{1} \* \\[-1mm]
            \cdots \* \vdots \* \cdots \\[-1mm]
            \* b_{n} \*
        }\ .
    \end{align*}
    \serial{2.}多重齐次性:
    \begin{align*}
        \determinant{ccc}{
            \* \vdots \* \\[-1mm]
            ka_{1} \* \cdots \* ka_{n} \\[-1mm]
            \* \vdots \*
        }=k\determinant{ccc}{
            \* \vdots \*\\[-1mm]
            a_{1} \* \cdots \* a_{n} \\[-1mm]
            \* \vdots \*
        }\ ,\\[-14mm]
    \end{align*}
    \begin{align*}
        \determinant{ccccc}{
            \* ka_{1} \* \\[-1mm]
            \cdots \* \vdots \* \cdots \\[-1mm]
            \* ka_{n} \*
        }=k\determinant{ccccc}{
            \* a_{1} \* \\[-1mm]
            \cdots \* \vdots \* \cdots \\[-1mm]
            \* a_{n} \*
        }\ .
    \end{align*}
    \serial{3.}交错性:
    \begin{align*}
        \determinant{ccc}{
            \* \vdots \* \\[-1mm]
            a_{1} \* \cdots \* a_{n} \\[-1mm]
            \* \vdots \* \\[-1mm]
            a_{1} \* \cdots \* a_{n} \\[-1mm]
            \* \vdots \*
        }=0\ ,\\[-14mm]
    \end{align*}
    \begin{align*}
        \determinant{ccccc}{
            \* a_{1} \* \* a_{1} \* \\[-1mm]
            \cdots \* \vdots \* \cdots \* \vdots \* \cdots \\[-1mm]
            \* a_{n} \* \* a_{n} \*
        }=0\ .
    \end{align*}
}
\Proof{\sentence{1}证明}{
    \[\resizebox{\textwidth}{!}{$
        \begin{aligned}
            \determinant{ccc}{
                a_{1,1} \* \cdots \* a_{1,n} \\[-1mm]
                \* \vdots \* \\[-1mm]
                a_{m-1,1} \* \cdots \* a_{m-1,n} \\[-1mm]
                a_{1}+b_{1} \* \cdots \* a_{n}+b_{n} \\[-1mm]
                a_{m+1,1} \* \cdots \* a_{m+1,n} \\[-1mm]
                \* \vdots \* \\[-1mm]
                a_{n,1} \* \cdots \* a_{n,n}
            }&=\sum_{\sigma\in S_{n}}\braI{\sgn\sigma\braI{a_{\sigma(m)}+b_{\sigma(m)}}\prod_{\substack{1\leqslant k\leqslant n\\k\ne m}}a_{k,\sigma\braI{k}\!}}\\[1mm]
            &=\sum_{\sigma\in S_{n}}\braI{\sgn\sigma\,a_{\sigma(m)}\prod_{\substack{1\leqslant k\leqslant n\\k\ne m}}a_{k,\sigma\braI{k}\!}}+\sum_{\sigma\in S_{n}}\braI{\sgn\sigma\,b_{\sigma(m)}\prod_{\substack{1\leqslant k\leqslant n\\k\ne m}}a_{k,\sigma\braI{k}\!}}\\[1mm]
            &=\determinant{ccc}{
                a_{1,1} \* \cdots \* a_{1,n} \\[-1mm]
                \* \vdots \* \\[-1mm]
                a_{m-1,1} \* \cdots \* a_{m-1,n} \\[-1mm]
                a_{1} \* \cdots \* a_{n} \\[-1mm]
                a_{m+1,1} \* \cdots \* a_{m+1,n} \\[-1mm]
                \* \vdots \* \\[-1mm]
                a_{n,1} \* \cdots \* a_{n,n}
            }+\determinant{ccc}{
                a_{1,1} \* \cdots \* a_{1,n} \\[-1mm]
                \* \vdots \* \\[-1mm]
                a_{m-1,1} \* \cdots \* a_{m-1,n} \\[-1mm]
                b_{1} \* \cdots \* b_{n} \\[-1mm]
                a_{m+1,1} \* \cdots \* a_{m+1,n} \\[-1mm]
                \* \vdots \* \\[-1mm]
                a_{n,1} \* \cdots \* a_{n,n}
            }\ ,
        \end{aligned}
    $}\]
    另一个等式可由完全相似的方式证出.\ProofEndT
}
\Proof{\sentence{2}证明}{
    \[\resizebox{\textwidth}{!}{$
        \begin{aligned}
        \determinant{ccccccc}{
            a_{1,1} \* \* a_{1,m-1} \* ka_{1} \* a_{1,m+1} \* \* a_{1,n} \\[-1mm]
            \vdots \* \cdots \* \vdots \* \vdots \* \vdots \* \cdots \* \vdots \\[-1mm]
            a_{n,1} \* \* a_{n,m-1} \* ka_{n} \* a_{n,m+1} \* \* a_{n,n}
        }&=\sum_{\sigma\in S_{n}}\braI{\sgn\sigma\,ka_{\sigma(m)}\prod_{\substack{1\leqslant k\leqslant n\\k\ne m}}a_{k,\sigma\braI{k}\!}}\\[1mm]
        &=k\sum_{\sigma\in S_{n}}\braI{\sgn\sigma\,a_{\sigma(m)}\prod_{\substack{1\leqslant k\leqslant n\\k\ne m}}a_{k,\sigma\braI{k}\!}}\\[1mm]
        &=k\determinant{ccccccc}{
            a_{1,1} \* \* a_{1,m-1} \* a_{1} \* a_{1,m+1} \* \* a_{1,n} \\[-1mm]
            \vdots \* \cdots \* \vdots \* \vdots \* \vdots \* \cdots \* \vdots \\[-1mm]
            a_{n,1} \* \* a_{n,m-1} \* a_{n} \* a_{n,m+1} \* \* a_{n,n}
        }\ ,
        \end{aligned}
    $}\]
    另一个等式可由完全相似的方式证出.\ProofEndT
}

\newpage

\Proof{\sentence{3}证明}{
    \par 对于任意给定的正整数 $p\leqslant q\leqslant n$ , 我们定义关于 $p$ 和 $q$ 的对换映射 $\mathcal{A}:S_{n}\to S_{n}$ 为
    \begin{align*}
        \braI{\mathcal{A}\sigma}\braI{k}:=\casesIII{\sigma\braI{q}}{k=p}{-2}{\sigma\braI{p}}{k=q}{-2}{\sigma\braI{k}}{k\ne p\Text{且}k\ne q}\ ,
    \end{align*}
    显然 $\mathcal{A}$ 是双射. 若 $\bm A\braI{p\and k}=\bm A\braI{q\and k}$ 对一切正整数 $k\leqslant n$ 成立, 则
    \begin{align*}
        \prod_{k=1}^{n}\bm A\braI{k\and\braI{\mathcal{A}\sigma}\braI{k}\!}&=\bm A\braI{p\and\braI{\mathcal{A}\sigma}\braI{p}\!}\bm A\braI{q\and\braI{\mathcal{A}\sigma}\braI{q}\!}\prod_{\substack{1\leqslant k\leqslant n\\k\ne p\text{且}k\ne q}}\bm A\braI{k\and\braI{\mathcal{A}\sigma}\braI{k}\!}\\[1mm]
        &=\bm A\braI{p\and\sigma\braI{q}\!}\bm A\braI{q\and\sigma\braI{p}\!}\prod_{\substack{1\leqslant k\leqslant n\\k\ne p\text{且}k\ne q}}\bm A\braI{k\and\sigma\braI{k}\!}\\[1mm]
        &=\bm A\braI{q\and\sigma\braI{q}\!}\bm A\braI{p\and\sigma\braI{p}\!}\prod_{\substack{1\leqslant k\leqslant n\\k\ne p\text{且}k\ne q}}\bm A\braI{k\and\sigma\braI{k}\!}\\[1mm]
        &=\prod_{k=1}^{n}\bm A\braI{k\and\sigma\braI{k}\!}\ ,
    \end{align*}
    于是
    \begin{align*}
        \det\bm A&=\frac{1}{2}\braI{\sum_{\sigma\in S_{n}}\sgn\sigma\prod_{k=1}^{n}\bm A\braI{k\and\sigma\braI{k}\!}+\sum_{\sigma\in S_{n}}\sgn\braI{\mathcal{A}\sigma}\prod_{k=1}^{n}\bm A\braI{k\and\braI{\mathcal{A}\sigma}\braI{k}\!}}\\[1mm]
        &=\frac{1}{2}\braI{\sum_{\sigma\in S_{n}}\sgn\sigma\prod_{k=1}^{n}\bm A\braI{k\and\sigma\braI{k}\!}+\sum_{\sigma\in S_{n}}\sgn\braI{\mathcal{A}\sigma}\prod_{k=1}^{n}\bm A\braI{k\and\sigma\braI{k}\!}}\\[1mm]
        &=\frac{1}{2}\sum_{\sigma\in S_{n}}\braI{\sgn\sigma+\sgn\braI{\mathcal{A}\sigma}\!}\prod_{k=1}^{n}\bm A\braI{k\and\sigma\braI{k}\!}\\[-1mm]
        &=0\ ,
    \end{align*}
    另一个等式可由完全相似的方式证出.\ProofEndT
}

\par 上述性质表明行列式作为映射
\begin{align*}
    \det_{\mathrm{column}}:\braI{\bfF^{n}}^{n}\to\bfF\ ,\ \braI{(a_{k}^{(1)})_{k=1}^{n}\and\cdots\and(a_{k}^{(n)})_{k=1}^{n}}\mapsto\determinant{ccc}{
        a_{1}^{(1)} \* \cdots \* a_{n}^{(1)} \\[-1mm]
        \vdots \* \ddots \* \vdots \\[-1mm]
        a_{1}^{(n)} \* \cdots \* a_{n}^{(n)}
    }
\end{align*}
和
\begin{align*}
    \det_{\mathrm{row}}:\braI{\bfF^{n}}^{n}\to\bfF\ ,\ \braI{(a_{1}^{(k)})_{k=1}^{n}\and\cdots\and(a_{n}^{(k)})_{k=1}^{n}}\mapsto\determinant{ccccc}{
        a_{1}^{(1)} \* \cdots \* a_{n}^{(1)} \\[-1mm]
        \vdots \* \ddots \* \vdots \\[-1mm]
        a_{1}^{(n)} \* \cdots \* a_{n}^{(n)}
    }
\end{align*}
是\link{线性空间}[多重线性型]{交错多重线性型}(交错$n$重线性型), 由此可以得到下面的诸多结论.

\Corollary{行列式的剪切不变性}{
    \begin{align*}
        \determinant{ccc}{
            \* \vdots \* \\[-1mm]
            a_{1} \* \cdots \* a_{n} \\[-1mm]
            \* \vdots \* \\[-1mm]
            b_{1} \* \cdots \* b_{n} \\[-1mm]
            \* \vdots \*
        }=\determinant{ccc}{
            \* \vdots \* \\[-1mm]
            a_{1} \* \cdots \* a_{n} \\[-1mm]
            \* \vdots \* \\[-1mm]
            b_{1}+ca_{1} \* \cdots \* b_{n}+ca_{n} \\[-1mm]
            \* \vdots \*
        }\ ,\\[-14mm]
    \end{align*}
    \begin{align*}
        \determinant{ccccc}{
            \* a_{1} \* \* b_{1} \* \\[-1mm]
            \cdots \* \vdots \* \cdots \* \vdots \* \cdots \\[-1mm]
            \* a_{n} \* \* b_{n} \*
        }=\determinant{ccccccc}{
            \* a_{1} \* \* b_{1}+ca_{1} \* \\[-1mm]
            \cdots \* \vdots \* \cdots \* \vdots \* \cdots \\[-1mm]
            \* a_{n} \* \* b_{n}+ca_{n} \*
        }\ .
    \end{align*}
}

\vspace{15pt}

\vspace{-4pt}

\Corollary{行列式的反对称性}{
    \begin{align*}
        \determinant{ccc}{
            \* \vdots \* \\[-1mm]
            a_{1} \* \cdots \* a_{n} \\[-1mm]
            \* \vdots \* \\[-1mm]
            b_{1} \* \cdots \* b_{n} \\[-1mm]
            \* \vdots \*
        }=-\determinant{ccc}{
            \* \vdots \* \\[-1mm]
            b_{1} \* \cdots \* b_{n} \\[-1mm]
            \* \vdots \* \\[-1mm]
            a_{1} \* \cdots \* a_{n} \\[-1mm]
            \* \vdots \*
        }\ ,\\[-14mm]
    \end{align*}
    \begin{align*}
        \determinant{ccccc}{
            \* a_{1} \* \* b_{1} \* \\[-1mm]
            \cdots \* \vdots \* \cdots \* \vdots \* \cdots \\[-1mm]
            \* a_{n} \* \* b_{n} \*
        }=-\determinant{ccccc}{
            \* b_{1} \* \* a_{1} \* \\[-1mm]
            \cdots \* \vdots \* \cdots \* \vdots \* \cdots \\[-1mm]
            \* b_{n} \* \* a_{n} \*
        }\ .
    \end{align*}
}

\vspace{15pt}

\vspace{-4pt}

\Corollary{奇异矩阵的行列式}{
    $\bm A$ 是奇异矩阵当且仅当 $\det\bm A=0$ .
}

\par 从 $\bm I_{n}$ 的最后一行出发, 依次向上逐行应用行列式的\link{域}[行列式]{行列式的交错多重线性}(多重齐次性)和\link{域}[行列式]{行列式的剪切不变性}(剪切不变性)即可得任意上三角矩阵的行列式.

\newpage

\Corollary{上三角矩阵的行列式}{
    \begin{align*}
        \determinant{cccccc}{
            \lambda_{1} \* * \* \* \cdots \* \* * \\[-1mm]
            0 \* \lambda_{2} \* \\[-1mm]
            \* \* \ddots \* \* \* \vdots \\[-1mm]
            \vdots \* \* \* \ddots \* \* \\[-1mm]
            \* \* \* \* \lambda_{n-1} \* * \\[-1mm]
            0 \* \* \cdots \* \* 0 \* \lambda_{n}
        }=\prod_{k=1}^{n}\lambda_{k}\ .
    \end{align*}
}

\vspace{-4pt}

\Theorem{转置矩阵的行列式}{
    $\det\bm A^{\rmT}=\det\bm A$ .
}
\Proof{证明}{
    \begin{align*}
        \det\bm A^{\rmT}&=\sum_{\sigma\in S_{n}}\sgn\sigma\prod_{k=1}^{n}\bm A^{\rmT}\braI{k\and\sigma\braI{k}\!}\\[1mm]
        &=\sum_{\sigma\in S_{n}}\sgn\sigma\prod_{k=1}^{n}\bm A\braI{\sigma\braI{k}\and k}\\[1mm]
        &=\sum_{\sigma\in S_{n}}\sgn\sigma\prod_{k=1}^{n}\bm A\braI{k\and\sigma^{-1}\braI{k}\!}\\[1mm]
        &=\sum_{\sigma\in S_{n}}\sgn\sigma\prod_{k=1}^{n}\bm A\braI{k\and\sigma\braI{k}\!}\\[1mm]
        &=\det\bm A\ .\ProofEndF
    \end{align*}
}

\Theorem{行列式的可乘性}{
    $\det\braI{\bm A\bm B}=\det\bm A\,\det\bm B$ .
}
\Proof{证明}{
    \begin{align*}
        \det\braI{\bm A\bm B}&=\sum_{\sigma\in S_{n}}\sgn\sigma\prod_{k=1}^{n}\bm A\bm B\braI{k\and\sigma\braI{k}\!}\\[1mm]
        &=\sum_{\sigma\in S_{n}}\sgn\sigma\prod_{k=1}^{n}\sum_{l=1}^{n}\bm A\braI{k\and l}\bm B\braI{l\and\sigma\braI{k}\!}\\[1mm]
        &=\sum_{\sigma\in S_{n}}\sgn\sigma\sum_{1\leqslant l_{1},...,l_{n}\leqslant n}\prod_{k=1}^{n}\bm A\braI{k\and l_{k}}\bm B\braI{l_{k}\and\sigma\braI{k}\!}\\[1mm]
        &=\sum_{1\leqslant l_{1},...,l_{n}\leqslant n}\prod_{k=1}^{n}\bm A\braI{k\and l_{k}}\sum_{\sigma\in S_{n}}\sgn\sigma\prod_{k=1}^{n}\bm B\braI{l_{k}\and\sigma\braI{k}\!}\ ,
    \end{align*}

\newpage

    \noindent 考虑矩阵
    \begin{align*}
        \bm B'=\matrix{ccc}{
            \bm B\braI{l_{1}\and 1} \* \cdots \* \bm B\braI{l_{1}\and n} \\[-1mm]
            \vdots \* \ddots \* \vdots \\[-1mm]
            \bm B\braI{l_{n}\and 1} \* \cdots \* \bm B\braI{l_{n}\and n}
        }\ ,
    \end{align*}
    则上面的结果可写作
    \begin{align*}
        \det\braI{\bm A\bm B}=\sum_{1\leqslant l_{1},...,l_{n}\leqslant n}\det\bm B'\prod_{k=1}^{n}\bm A\braI{k\and l_{k}}\ ,
    \end{align*}
    我们可以要求上述 $l_{k}$ 各不相同, 否则就有 $\det\bm B'=0$ , 因此我们可以进一步将 $l_{1}\and\cdots\and l_{n}$ 视作 $\braIII{1\and\cdots\and n}$ 上的某个置换 $\tau$ , 此时又有 $\det\bm B'=\sgn\tau\det\bm B$ , 于是
    \begin{align*}
        \det\braI{\bm A\bm B}&=\sum_{\tau\in S_{n}}\det\bm B'\prod_{k=1}^{n}\bm A\braI{k\and\tau\braI{k}\!}\\[1mm]
        &=\det\bm B\sum_{\tau\in S_{n}}\sgn\tau\prod_{k=1}^{n}\bm A\braI{k\and\tau\braI{k}\!}\\[1mm]
        &=\det\bm A\,\det\bm B\ .\ProofEndF
    \end{align*}
}

\Theorem{逆矩阵的行列式}{
    若 $\bm A$ 可逆, 则 $\det\braI{\bm A^{-1}}=\braI{\det\bm A}^{-1}$ .
}
\Proof{证明}{
    \begin{align*}
        \det\braI{\bm A^{-1}}=\frac{\det\braI{\bm A\bm A^{-1}}}{\det\bm A}=\frac{\det\bm I}{\det\bm A}=\frac{1}{\det\bm A}\ .\ProofEndF
    \end{align*}
}

\newpage
